\documentclass[11pt]{article}
%\usepackage{amsfonts}
\usepackage{amsmath}
\usepackage{fancybox}%,times}
\usepackage{graphicx,psfrag,epsf}
%\usepackage{amsmath}
\usepackage{enumerate}
\usepackage{graphicx,psfrag}
\usepackage{multirow}
\usepackage{epsfig}
%\usepackage{rotating}
\usepackage{subfigure}
\usepackage{theorem}
\usepackage{natbib,psfrag}
\usepackage{tikz}
\usepackage{xcolor}
\usepackage{kotex}
\newcommand{\blind}{0}
\usepackage{graphicx}
\DeclareGraphicsExtensions{.pdf,.png,.jpg}

\addtolength{\oddsidemargin}{-.75in}%
\addtolength{\evensidemargin}{-.75in}%
\addtolength{\textwidth}{1.5in}%
\addtolength{\textheight}{1.3in}%
%\addtolength{\topmargin}{-.6in}%
\addtolength{\topmargin}{-.8in}%

%\theoremstyle{break}
\newtheorem{The}{Theorem}
\newtheorem{Def}{Definition}
\newtheorem{Pro}{Proposition}
\newtheorem{Lem}{Lemma}
\newtheorem{Cor}{Corollary}
\newtheorem{asp}{Assumption}


\renewcommand{\thefootnote}{\arabic{footnote}}
%\renewcommand{\thefootnote}{\alph{footnote}}
%\renewcommand{\thefootnote}{\roman{footnote}}
%\renewcommand{\thefootnote}{\fnsymbol{footnote}}

\begin{document}


%\bibliographystyle{natbib}

\newcommand{\Ito}{$It\hat{o}$'$s~Lemma$}

\newcommand\ind{\stackrel{\rm ind}{\sim}}
\newcommand\iid{\stackrel{\rm iid}{\sim}}
\renewcommand\c{\mathbf{c}}
\newcommand\y{\mathbf{y}}
\newcommand\z{\mathbf{z}}
\renewcommand\P{\mathbf{P}}
\newcommand\W{\mathbf{W}}
\newcommand\X{\mathbf{X}}
\newcommand\Y{\mathbf{Y}}
\newcommand\Z{\mathbf{Z}}
\newcommand\J{{\cal J}}
\newcommand\B{{\cal B}}
\newcommand\K{{\cal K}}
\newcommand\N{{\rm N}}
\newcommand\bs{\boldsymbol}
\newcommand\bth{\bs\theta}
\newcommand\bbe{\bs\beta}
\renewcommand\*{^\star}

\def\spacingset#1{\renewcommand{\baselinestretch}%
{#1}\small\normalsize} \spacingset{1}


%%%%%%%%%%%%%%%%%%%%%%%%%%%%%%%%%%%%%%%%%%%%%%%%%%%%%%%%%%%%%%%%%%%%%%%%%%%%%%

  \bigskip
  \bigskip
  \bigskip
  \begin{center}
    {\LARGE\bf Jan 09, 2019 }
  \end{center}
  \medskip

%\begin{abstract}
%\end{abstract}

%\noindent%
%{\it Key Words:}  AECM algorithm; Astrophysical data analysis;
%ECME algorithm; Incompatible Gibbs sampler; Marginal data
%augmentation; Multiple imputation; Spectral analysis

\spacingset{1.45}



\section{Model of peer effects with an endogenous network} 

\subsection{Algebric} 

  \begin{align}
    y_{i} = \beta_{1}^0 \sum\limits_{j=1, j \neq i}^N g_{ij,N}y_{j} + \mathbf{x'}_{1i}\beta_{2}^0 + 	\left(\sum\limits_{j=1, j \neq i}^N g_{ij,N}\mathbf{x}_{1i} \right) \beta_{3}^0 + \nu_{i}
  \end{align}


 Suppose $d_{ij,N}$ are the observed links among individuals $i \in {1, \dots ,N}$ , $d_{ij,N} = 1$ if $i$ and $j$ are directly connected and 0 otherwise.
Where $\mathbf{x'}_{1i}$ are observed individual characteristics that affect the outcome $y_{i}$, $\nu_{i}$ are unobserved individual characteristics, and
$ g_{ij,N} = 0$ if $i =j$ , $g_{ij,N} = \frac{d_{ij,N}}{\sum\limits_{j \neq i}d_{ij,N}}$ otherwise which is the weight of the peer effect.

$\beta_{1}^0$ captures the endogenous social effect, and $\beta_{3}^0$ measures the exogenous social effect.
\subsection{Matrix} 

$ \mathbf{D}_{N}$ be the $(N \times N)$  adjacency matrix of the network whose $(i, j)^{th}$ element is $d_{ij,N}$.
$ \mathbf{G}_{N}$ be the $(N \times N)$  adjacency matrix of the network whose $(i, j)^{th}$ element is $g_{ij,N}$.
$ \mathbf{X}_{1N} = (\mathbf{x'}_{11}, \dots ,\mathbf{x'}_{1N})' $, 
$ \mathbf{y}_{N} = (y_{1}, \dots ,y_{N})' $ ,
$ \boldsymbol{\nu}_{N} = (\nu_{1}, \dots ,\nu_{N})' $
  \begin{align}
\mathbf{y}_{N} = \beta_{1}\mathbf{G}_{N}\mathbf{y}_{N} + \mathbf{X}_{1N}\beta_{2} + \mathbf{G}_{N}\mathbf{X}_{1N}\beta_{3} + \boldsymbol{\nu}_{N} \\
\mathbf{y}_{N} = (\mathbf{I}_{N} - \beta_{1}\mathbf{G}_{N})^{-1}(\mathbf{X}_{1N}\beta_{2} + \mathbf{G}_{N}\mathbf{X}_{1N}\beta_{3} + \boldsymbol{\nu}_{N} )
  \end{align}
\\
\\

assuming that the peer group (or the network) is exogenous $E[\nu_{i}|\mathbf{X}_{1N} , \mathbf{G}_{N} ] = 0 $. the fact that the regressor $\sum\limits_{j=1}^N g_{ij,N}y_{j}$ is correlated with the error term $\nu_{i}$. For example, if $\nu_{i} \sim i.i.d.(0, \sigma^2)$, it is true that
  \begin{align}
E[(\mathbf{G}_{N}\mathbf{y}_{N})'\boldsymbol{\nu}_{N}] = E[(\mathbf{G}_{N}(\mathbf{I}_{N} - \beta_{1}\mathbf{G}_{N})^{-1}(\mathbf{X}_{1N}\beta_{2} + \mathbf{G}_{N}\mathbf{X}_{1N}\beta_{3} + \boldsymbol{\nu}_{N} ))'\boldsymbol{\nu}_{N}]\\
= E[(\mathbf{G}_{N}(\mathbf{I}_{N} - \beta_{1}\mathbf{G}_{N})^{-1}\boldsymbol{\nu}_{N})'\boldsymbol{\nu}_{N}] = \sigma_{0}tr(\mathbf{G}_{N}(\mathbf{I}_{N} - \beta_{1}\mathbf{G}_{N})^{-1}) \neq 0
  \end{align}
One of the widely used estimation methods is the Instrumental Variable (IV) approach based on using $\mathbf{G}_{N}^2\mathbf{X}_{1N}$ as the IV of the endogenous regressor $\mathbf{G}_{N}\mathbf{y}_{N}$. Then, the natural estimator is the Two-Stage Least Squares (2SLS) estimator
  \begin{align}
\hat{\beta}_{N}^{2SLS} = (\mathbf{W}_{N}'\mathbf{Z}_{N}(\mathbf{Z}_{N}'\mathbf{Z}_{N})^{-1}\mathbf{Z}_{N}'\mathbf{W}_{N})\mathbf{W}_{N}'\mathbf{Z}_{N}(\mathbf{Z}_{N}'\mathbf{Z}_{N})^{-1}\mathbf{Z}_{N}'\mathbf{y}_{N}
  \end{align}
where $\mathbf{W}_{N} = [\mathbf{G}_{N}\mathbf{y}_{N},\mathbf{X}_{1N} ,\mathbf{G}_{N}\mathbf{X}_{1N} ]$ and 
$\mathbf{Z}_{N} = [\mathbf{X}_{1N} , \mathbf{G}_{N}\mathbf{X}_{1N} , \mathbf{G}_{N}^2\mathbf{X}_{1N}]$ we asuume $\beta_{2}^0 \neq 0$

When the network matrix is endogenous, $E[\mathbf{G}_{N} | \boldsymbol{\nu}_{N} ] \neq 0$, the procedure is no longer
valid since the IV matrix $\mathbf{Z}_{N} = [\mathbf{X}_{1N} , \mathbf{G}_{N}\mathbf{X}_{1N} , \mathbf{G}_{N}^2\mathbf{X}_{1N}]$ is correlated with the error term $\boldsymbol{\nu}_{N}$  Specifically, the validity of the 2SLS estimator depends on the orthogonality condition
$E[\boldsymbol{\nu}_{N} |\mathbf{Z}_{N} ] = 0$, which is implied if $E[\boldsymbol{\nu}_{N} |\mathbf{X}_{1N} , \mathbf{D}_{N} ] = 0$. However, it does not hold if the network $\mathbf{D}_{N}$ (or equivalently, the network $\mathbf{G}_{N}$ ) is correlated with $\boldsymbol{\nu}_{N}$.

\section{Endogenous Network Formation and Identification of peer effects}
\subsection{Formation}


Let $\mathbf{x}_{2i}$ be a vector of observable characteristics of individual $i$, and let $\mathbf{x}_{i} = \mathbf{x1}_{i}\cup \mathbf{x}_{2i}$. 
Define $\mathbf{X}_{2N}$ analogously to $\mathbf{X}_{1N}$ and let $\mathbf{X}_{N} = \mathbf{X}_{1N} \cup \mathbf{X2}_{N}$ .
We introduce $a_{i}$, a scalar unobserved characteristic of individual $i$, which is treated as an
individual fixed-effect, and hence, might be correlated with $\mathbf{x}_{i}$. We denote the vector of
individual unobserved characteristics by $\mathbf{a}_{N} = (a_{1}, a_{2}, \dots , a_{N} )$′. Individuals are connected
by an undirected network $\mathbf{D}_{N}$ , with the $(i, j)^{th}$ element $d_{ij,N} = 1$ if $i$ and $j$ are directly
connected and 0 otherwise. We assume the network to be undirected, $d_{ij,N} = d_{ji,N}$ , and
assume $d_{ii,N} = 0$ for all $i$, following the convention. In this case, there are $n = \binom{N}{2}$ dyads.
Let $\mathbf{t}_{ij}$ denote an $l_{T} \times 1$  vector of dyad-specific characteristics of dyad $ij$, and we assume
that $\mathbf{t}_{ij} = t(\mathbf{x}_{2i}, \mathbf{x}_{2j} )$.
  \begin{align}
d_{ij,N} = I(g(t(\mathbf{x}_{2i}, \mathbf{x}_{2j} ), a_{i}, a_{j} ) - u_{ij} \geq 0) \\
d_{ij,N} = I(t(\mathbf{x}_{2i}, \mathbf{x}_{2j} )'\lambda + a_{i} + a_{j} - u_{ij} > 0)
  \end{align}
where $I()$ is an indicator function.



\end{document}
