 \documentclass[11pt]{article}
%\usepackage{amsfonts}
\usepackage{amsmath}
\usepackage{fancybox}%,times}
\usepackage{graphicx,psfrag,epsf}
%\usepackage{amsmath}
\usepackage{enumerate}
\usepackage{graphicx,psfrag}
\usepackage{multirow}
\usepackage{epsfig}
%\usepackage{rotating}
\usepackage{subfigure}
\usepackage{theorem}
\usepackage{natbib,psfrag}
\usepackage{tikz}
\usepackage{xcolor}
\usepackage{kotex}


\newcommand{\blind}{0}
\usepackage{graphicx}
\DeclareGraphicsExtensions{.pdf,.png,.jpg}

\addtolength{\oddsidemargin}{-.75in}%
\addtolength{\evensidemargin}{-.75in}%
\addtolength{\textwidth}{1.5in}%
\addtolength{\textheight}{1.3in}%
%\addtolength{\topmargin}{-.6in}%
\addtolength{\topmargin}{-.8in}%

%\theoremstyle{break}
\newtheorem{The}{Theorem}
\newtheorem{Def}{Definition}
\newtheorem{Pro}{Proposition}
\newtheorem{Lem}{Lemma}
\newtheorem{Cor}{Corollary}
\newtheorem{asp}{Assumption}


\renewcommand{\thefootnote}{\arabic{footnote}}
%\renewcommand{\thefootnote}{\alph{footnote}}
%\renewcommand{\thefootnote}{\roman{footnote}}
%\renewcommand{\thefootnote}{\fnsymbol{footnote}}

\begin{document}


%\bibliographystyle{natbib}

\newcommand{\Ito}{$It\hat{o}$'$s~Lemma$}

\newcommand\ind{\stackrel{\rm ind}{\sim}}
\newcommand\iid{\stackrel{\rm iid}{\sim}}
\renewcommand\c{\mathbf{c}}
\newcommand\y{\mathbf{y}}
\newcommand\z{\mathbf{z}}
\renewcommand\P{\mathbf{P}}
\newcommand\W{\mathbf{W}}
\newcommand\X{\mathbf{X}}
\newcommand\Y{\mathbf{Y}}
\newcommand\Z{\mathbf{Z}}
\newcommand\J{{\cal J}}
\newcommand\B{{\cal B}}
\newcommand\K{{\cal K}}
\newcommand\N{{\rm N}}
\newcommand\bs{\boldsymbol}
\newcommand\bth{\bs\theta}
\newcommand\bbe{\bs\beta}
\renewcommand\*{^\star}

\def\spacingset#1{\renewcommand{\baselinestretch}%
{#1}\small\normalsize} \spacingset{1}


%%%%%%%%%%%%%%%%%%%%%%%%%%%%%%%%%%%%%%%%%%%%%%%%%%%%%%%%%%%%%%%%%%%%%%%%%%%%%%

  \bigskip
  \bigskip
  \bigskip
  \begin{center}
    {\LARGE\bf March 14, 2019 }
  \end{center}
  \medskip

%\begin{abstract}
%\end{abstract}

%\noindent%
%{\it Key Words:}  AECM algorithm; Astrophysical data analysis;
%ECME algorithm; Incompatible Gibbs sampler; Marginal data
%augmentation; Multiple imputation; Spectral analysis

\spacingset{1.45}


\section{Introduction}
The Metropilis-Hastings(MH) algorithm simulates from a probability distribution by making use of the full joint density function and proposal distributions

\begin{itemize}
	\item find $p(x|y)$ where $\pi(x)p(y|x) = \pi(y)p(x|y)$
	\item under some condition $p(x|y)$ converge to $\pi(x)$
\end{itemize}

\begin{align*}
p(x|y) &= \alpha(x|y) q(x|y)
\end{align*}

\begin{itemize}
	\item $q(x|y)$ is proposal distribution.
	\item $\alpha(x|y)$ is acceptance probability
\end{itemize}
\begin{align*}
\pi(x)p(y|x) &= \pi(y)q(x|y) \\
\pi(x) \alpha(y|x) q(y|x)& = \pi(y) \alpha(x|y) q(x|y)
\end{align*}
if $\pi(x)q(y|x) > \pi(y) q(x|y)$

\begin{align*}
\alpha(x|y) &= 1 \\
\alpha(y|x) &= \frac{\pi(y)  q(x|y)}{\pi(x)q(y|x)}
\end{align*}

\section{MH}
Initailize $x^{(0)} \sim q(x)$ and repeat the Process during the iteration
\begin{itemize}
	\item Propose $x^{cand} \sim q(x^{(i)}| x^{(i-1)})$
	\item Aceeptance Prob $\alpha(x^{(cand)}| x^{(i-1)}) = min(1, \frac{\pi(x^{(cand)})  q(x^{(i-1)}|x^{(cand)})}{\pi(x^{(i-1)}) q(x^{(cand)}|x^{(i-1)})})$
	\item $u \sim Unif(u ; 0,1)$
	\item if $ u > \alpha $ Accept the Prob $x^{(i)} = x^{(cand)}$
	\item else reject the Prob $x^{(i)} = x^{(i-1)}$
\end{itemize}

\end{document} 
