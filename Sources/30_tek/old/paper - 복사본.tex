\documentclass[11pt]{article}
%\usepackage{amsfonts}
\usepackage{amsmath}
\usepackage{fancybox}%,times}
\usepackage{graphicx,psfrag,epsf}
%\usepackage{amsmath}
\usepackage{enumerate}
\usepackage{graphicx,psfrag}
\usepackage{multirow}
\usepackage{epsfig}
%\usepackage{rotating}
\usepackage{subfigure}
\usepackage{theorem}
\usepackage{natbib,psfrag}
\usepackage{tikz}
\usepackage{xcolor}
\usepackage{kotex}
\newcommand{\blind}{0}
\usepackage{graphicx}
\DeclareGraphicsExtensions{.pdf,.png,.jpg}

\addtolength{\oddsidemargin}{-.75in}%
\addtolength{\evensidemargin}{-.75in}%
\addtolength{\textwidth}{1.5in}%
\addtolength{\textheight}{1.3in}%
%\addtolength{\topmargin}{-.6in}%
\addtolength{\topmargin}{-.8in}%

%\theoremstyle{break}
\newtheorem{The}{Theorem}
\newtheorem{Def}{Definition}
\newtheorem{Pro}{Proposition}
\newtheorem{Lem}{Lemma}
\newtheorem{Cor}{Corollary}
\newtheorem{asp}{Assumption}


\renewcommand{\thefootnote}{\arabic{footnote}}
%\renewcommand{\thefootnote}{\alph{footnote}}
%\renewcommand{\thefootnote}{\roman{footnote}}
%\renewcommand{\thefootnote}{\fnsymbol{footnote}}

\begin{document}


%\bibliographystyle{natbib}

\newcommand{\Ito}{$It\hat{o}$'$s~Lemma$}

\newcommand\ind{\stackrel{\rm ind}{\sim}}
\newcommand\iid{\stackrel{\rm iid}{\sim}}
\renewcommand\c{\mathbf{c}}
\newcommand\y{\mathbf{y}}
\newcommand\z{\mathbf{z}}
\renewcommand\P{\mathbf{P}}
\newcommand\W{\mathbf{W}}
\newcommand\X{\mathbf{X}}
\newcommand\Y{\mathbf{Y}}
\newcommand\Z{\mathbf{Z}}
\newcommand\J{{\cal J}}
\newcommand\B{{\cal B}}
\newcommand\K{{\cal K}}
\newcommand\N{{\rm N}}
\newcommand\bs{\boldsymbol}
\newcommand\bth{\bs\theta}
\newcommand\bbe{\bs\beta}
\renewcommand\*{^\star}

\def\spacingset#1{\renewcommand{\baselinestretch}%
{#1}\small\normalsize} \spacingset{1}


%%%%%%%%%%%%%%%%%%%%%%%%%%%%%%%%%%%%%%%%%%%%%%%%%%%%%%%%%%%%%%%%%%%%%%%%%%%%%%

  \bigskip
  \bigskip
  \bigskip
  \begin{center}
    {\LARGE\bf Nov 05, 2018 }
  \end{center}
  \medskip

%\begin{abstract}
%\end{abstract}

%\noindent%
%{\it Key Words:}  AECM algorithm; Astrophysical data analysis;
%ECME algorithm; Incompatible Gibbs sampler; Marginal data
%augmentation; Multiple imputation; Spectral analysis

\spacingset{1.45}













\section{Ordinary Least Square}
Ordinary Least Square의 경우 오차의 평균이 0, 오차의 분산이 등분산, 오차가 서로 uncorrelated 일 경우 선형 모형 중에서 최적의 모델이다 (BLUE) 이때 오차의 분포가 정규분포라면 MLE와 동일한 결과를 보여준다

OLS가 아닌 다른 방법을 사용하는 이유는 변수의 개수 p보다 관측치의 개수 n이 충분히 많지 않은 경우 정확한 예측을 하지 못하는 문제가 발생하기 때문이다. 또한 실제 $\beta$값중에서 0이 되어야 하는 $\beta$ 대해서도 0에 가까운 값으로 예측 할 뿐 실제 0으로 예측하지 못하는 문제가 발생한다.

\section{Ridge regression}
  \begin{align}
    \hat{\beta}^{ridge}
  \end{align}


\end{document}
