
% Default to the notebook output style

    


% Inherit from the specified cell style.




    
\documentclass[11pt]{article}

    
    
    \usepackage[T1]{fontenc}
    % Nicer default font (+ math font) than Computer Modern for most use cases
    \usepackage{mathpazo}

    % Basic figure setup, for now with no caption control since it's done
    % automatically by Pandoc (which extracts ![](path) syntax from Markdown).
    \usepackage{graphicx}
    % We will generate all images so they have a width \maxwidth. This means
    % that they will get their normal width if they fit onto the page, but
    % are scaled down if they would overflow the margins.
    \makeatletter
    \def\maxwidth{\ifdim\Gin@nat@width>\linewidth\linewidth
    \else\Gin@nat@width\fi}
    \makeatother
    \let\Oldincludegraphics\includegraphics
    % Set max figure width to be 80% of text width, for now hardcoded.
    \renewcommand{\includegraphics}[1]{\Oldincludegraphics[width=.8\maxwidth]{#1}}
    % Ensure that by default, figures have no caption (until we provide a
    % proper Figure object with a Caption API and a way to capture that
    % in the conversion process - todo).
    \usepackage{caption}
    \DeclareCaptionLabelFormat{nolabel}{}
    \captionsetup{labelformat=nolabel}

    \usepackage{adjustbox} % Used to constrain images to a maximum size 
    \usepackage{xcolor} % Allow colors to be defined
    \usepackage{enumerate} % Needed for markdown enumerations to work
    \usepackage{geometry} % Used to adjust the document margins
    \usepackage{amsmath} % Equations
    \usepackage{amssymb} % Equations
    \usepackage{textcomp} % defines textquotesingle
    % Hack from http://tex.stackexchange.com/a/47451/13684:
    \AtBeginDocument{%
        \def\PYZsq{\textquotesingle}% Upright quotes in Pygmentized code
    }
    \usepackage{upquote} % Upright quotes for verbatim code
    \usepackage{eurosym} % defines \euro
    \usepackage[mathletters]{ucs} % Extended unicode (utf-8) support
    \usepackage[utf8x]{inputenc} % Allow utf-8 characters in the tex document
    \usepackage{fancyvrb} % verbatim replacement that allows latex
    \usepackage{grffile} % extends the file name processing of package graphics 
                         % to support a larger range 
    % The hyperref package gives us a pdf with properly built
    % internal navigation ('pdf bookmarks' for the table of contents,
    % internal cross-reference links, web links for URLs, etc.)
    \usepackage{hyperref}
    \usepackage{longtable} % longtable support required by pandoc >1.10
    \usepackage{booktabs}  % table support for pandoc > 1.12.2
    \usepackage[inline]{enumitem} % IRkernel/repr support (it uses the enumerate* environment)
    \usepackage[normalem]{ulem} % ulem is needed to support strikethroughs (\sout)
                                % normalem makes italics be italics, not underlines
    

    
    
    % Colors for the hyperref package
    \definecolor{urlcolor}{rgb}{0,.145,.698}
    \definecolor{linkcolor}{rgb}{.71,0.21,0.01}
    \definecolor{citecolor}{rgb}{.12,.54,.11}

    % ANSI colors
    \definecolor{ansi-black}{HTML}{3E424D}
    \definecolor{ansi-black-intense}{HTML}{282C36}
    \definecolor{ansi-red}{HTML}{E75C58}
    \definecolor{ansi-red-intense}{HTML}{B22B31}
    \definecolor{ansi-green}{HTML}{00A250}
    \definecolor{ansi-green-intense}{HTML}{007427}
    \definecolor{ansi-yellow}{HTML}{DDB62B}
    \definecolor{ansi-yellow-intense}{HTML}{B27D12}
    \definecolor{ansi-blue}{HTML}{208FFB}
    \definecolor{ansi-blue-intense}{HTML}{0065CA}
    \definecolor{ansi-magenta}{HTML}{D160C4}
    \definecolor{ansi-magenta-intense}{HTML}{A03196}
    \definecolor{ansi-cyan}{HTML}{60C6C8}
    \definecolor{ansi-cyan-intense}{HTML}{258F8F}
    \definecolor{ansi-white}{HTML}{C5C1B4}
    \definecolor{ansi-white-intense}{HTML}{A1A6B2}

    % commands and environments needed by pandoc snippets
    % extracted from the output of `pandoc -s`
    \providecommand{\tightlist}{%
      \setlength{\itemsep}{0pt}\setlength{\parskip}{0pt}}
    \DefineVerbatimEnvironment{Highlighting}{Verbatim}{commandchars=\\\{\}}
    % Add ',fontsize=\small' for more characters per line
    \newenvironment{Shaded}{}{}
    \newcommand{\KeywordTok}[1]{\textcolor[rgb]{0.00,0.44,0.13}{\textbf{{#1}}}}
    \newcommand{\DataTypeTok}[1]{\textcolor[rgb]{0.56,0.13,0.00}{{#1}}}
    \newcommand{\DecValTok}[1]{\textcolor[rgb]{0.25,0.63,0.44}{{#1}}}
    \newcommand{\BaseNTok}[1]{\textcolor[rgb]{0.25,0.63,0.44}{{#1}}}
    \newcommand{\FloatTok}[1]{\textcolor[rgb]{0.25,0.63,0.44}{{#1}}}
    \newcommand{\CharTok}[1]{\textcolor[rgb]{0.25,0.44,0.63}{{#1}}}
    \newcommand{\StringTok}[1]{\textcolor[rgb]{0.25,0.44,0.63}{{#1}}}
    \newcommand{\CommentTok}[1]{\textcolor[rgb]{0.38,0.63,0.69}{\textit{{#1}}}}
    \newcommand{\OtherTok}[1]{\textcolor[rgb]{0.00,0.44,0.13}{{#1}}}
    \newcommand{\AlertTok}[1]{\textcolor[rgb]{1.00,0.00,0.00}{\textbf{{#1}}}}
    \newcommand{\FunctionTok}[1]{\textcolor[rgb]{0.02,0.16,0.49}{{#1}}}
    \newcommand{\RegionMarkerTok}[1]{{#1}}
    \newcommand{\ErrorTok}[1]{\textcolor[rgb]{1.00,0.00,0.00}{\textbf{{#1}}}}
    \newcommand{\NormalTok}[1]{{#1}}
    
    % Additional commands for more recent versions of Pandoc
    \newcommand{\ConstantTok}[1]{\textcolor[rgb]{0.53,0.00,0.00}{{#1}}}
    \newcommand{\SpecialCharTok}[1]{\textcolor[rgb]{0.25,0.44,0.63}{{#1}}}
    \newcommand{\VerbatimStringTok}[1]{\textcolor[rgb]{0.25,0.44,0.63}{{#1}}}
    \newcommand{\SpecialStringTok}[1]{\textcolor[rgb]{0.73,0.40,0.53}{{#1}}}
    \newcommand{\ImportTok}[1]{{#1}}
    \newcommand{\DocumentationTok}[1]{\textcolor[rgb]{0.73,0.13,0.13}{\textit{{#1}}}}
    \newcommand{\AnnotationTok}[1]{\textcolor[rgb]{0.38,0.63,0.69}{\textbf{\textit{{#1}}}}}
    \newcommand{\CommentVarTok}[1]{\textcolor[rgb]{0.38,0.63,0.69}{\textbf{\textit{{#1}}}}}
    \newcommand{\VariableTok}[1]{\textcolor[rgb]{0.10,0.09,0.49}{{#1}}}
    \newcommand{\ControlFlowTok}[1]{\textcolor[rgb]{0.00,0.44,0.13}{\textbf{{#1}}}}
    \newcommand{\OperatorTok}[1]{\textcolor[rgb]{0.40,0.40,0.40}{{#1}}}
    \newcommand{\BuiltInTok}[1]{{#1}}
    \newcommand{\ExtensionTok}[1]{{#1}}
    \newcommand{\PreprocessorTok}[1]{\textcolor[rgb]{0.74,0.48,0.00}{{#1}}}
    \newcommand{\AttributeTok}[1]{\textcolor[rgb]{0.49,0.56,0.16}{{#1}}}
    \newcommand{\InformationTok}[1]{\textcolor[rgb]{0.38,0.63,0.69}{\textbf{\textit{{#1}}}}}
    \newcommand{\WarningTok}[1]{\textcolor[rgb]{0.38,0.63,0.69}{\textbf{\textit{{#1}}}}}
    
    
    % Define a nice break command that doesn't care if a line doesn't already
    % exist.
    \def\br{\hspace*{\fill} \\* }
    % Math Jax compatability definitions
    \def\gt{>}
    \def\lt{<}
    % Document parameters
    \title{2018321084???HW2}
    
    
    

    % Pygments definitions
    
\makeatletter
\def\PY@reset{\let\PY@it=\relax \let\PY@bf=\relax%
    \let\PY@ul=\relax \let\PY@tc=\relax%
    \let\PY@bc=\relax \let\PY@ff=\relax}
\def\PY@tok#1{\csname PY@tok@#1\endcsname}
\def\PY@toks#1+{\ifx\relax#1\empty\else%
    \PY@tok{#1}\expandafter\PY@toks\fi}
\def\PY@do#1{\PY@bc{\PY@tc{\PY@ul{%
    \PY@it{\PY@bf{\PY@ff{#1}}}}}}}
\def\PY#1#2{\PY@reset\PY@toks#1+\relax+\PY@do{#2}}

\expandafter\def\csname PY@tok@w\endcsname{\def\PY@tc##1{\textcolor[rgb]{0.73,0.73,0.73}{##1}}}
\expandafter\def\csname PY@tok@c\endcsname{\let\PY@it=\textit\def\PY@tc##1{\textcolor[rgb]{0.25,0.50,0.50}{##1}}}
\expandafter\def\csname PY@tok@cp\endcsname{\def\PY@tc##1{\textcolor[rgb]{0.74,0.48,0.00}{##1}}}
\expandafter\def\csname PY@tok@k\endcsname{\let\PY@bf=\textbf\def\PY@tc##1{\textcolor[rgb]{0.00,0.50,0.00}{##1}}}
\expandafter\def\csname PY@tok@kp\endcsname{\def\PY@tc##1{\textcolor[rgb]{0.00,0.50,0.00}{##1}}}
\expandafter\def\csname PY@tok@kt\endcsname{\def\PY@tc##1{\textcolor[rgb]{0.69,0.00,0.25}{##1}}}
\expandafter\def\csname PY@tok@o\endcsname{\def\PY@tc##1{\textcolor[rgb]{0.40,0.40,0.40}{##1}}}
\expandafter\def\csname PY@tok@ow\endcsname{\let\PY@bf=\textbf\def\PY@tc##1{\textcolor[rgb]{0.67,0.13,1.00}{##1}}}
\expandafter\def\csname PY@tok@nb\endcsname{\def\PY@tc##1{\textcolor[rgb]{0.00,0.50,0.00}{##1}}}
\expandafter\def\csname PY@tok@nf\endcsname{\def\PY@tc##1{\textcolor[rgb]{0.00,0.00,1.00}{##1}}}
\expandafter\def\csname PY@tok@nc\endcsname{\let\PY@bf=\textbf\def\PY@tc##1{\textcolor[rgb]{0.00,0.00,1.00}{##1}}}
\expandafter\def\csname PY@tok@nn\endcsname{\let\PY@bf=\textbf\def\PY@tc##1{\textcolor[rgb]{0.00,0.00,1.00}{##1}}}
\expandafter\def\csname PY@tok@ne\endcsname{\let\PY@bf=\textbf\def\PY@tc##1{\textcolor[rgb]{0.82,0.25,0.23}{##1}}}
\expandafter\def\csname PY@tok@nv\endcsname{\def\PY@tc##1{\textcolor[rgb]{0.10,0.09,0.49}{##1}}}
\expandafter\def\csname PY@tok@no\endcsname{\def\PY@tc##1{\textcolor[rgb]{0.53,0.00,0.00}{##1}}}
\expandafter\def\csname PY@tok@nl\endcsname{\def\PY@tc##1{\textcolor[rgb]{0.63,0.63,0.00}{##1}}}
\expandafter\def\csname PY@tok@ni\endcsname{\let\PY@bf=\textbf\def\PY@tc##1{\textcolor[rgb]{0.60,0.60,0.60}{##1}}}
\expandafter\def\csname PY@tok@na\endcsname{\def\PY@tc##1{\textcolor[rgb]{0.49,0.56,0.16}{##1}}}
\expandafter\def\csname PY@tok@nt\endcsname{\let\PY@bf=\textbf\def\PY@tc##1{\textcolor[rgb]{0.00,0.50,0.00}{##1}}}
\expandafter\def\csname PY@tok@nd\endcsname{\def\PY@tc##1{\textcolor[rgb]{0.67,0.13,1.00}{##1}}}
\expandafter\def\csname PY@tok@s\endcsname{\def\PY@tc##1{\textcolor[rgb]{0.73,0.13,0.13}{##1}}}
\expandafter\def\csname PY@tok@sd\endcsname{\let\PY@it=\textit\def\PY@tc##1{\textcolor[rgb]{0.73,0.13,0.13}{##1}}}
\expandafter\def\csname PY@tok@si\endcsname{\let\PY@bf=\textbf\def\PY@tc##1{\textcolor[rgb]{0.73,0.40,0.53}{##1}}}
\expandafter\def\csname PY@tok@se\endcsname{\let\PY@bf=\textbf\def\PY@tc##1{\textcolor[rgb]{0.73,0.40,0.13}{##1}}}
\expandafter\def\csname PY@tok@sr\endcsname{\def\PY@tc##1{\textcolor[rgb]{0.73,0.40,0.53}{##1}}}
\expandafter\def\csname PY@tok@ss\endcsname{\def\PY@tc##1{\textcolor[rgb]{0.10,0.09,0.49}{##1}}}
\expandafter\def\csname PY@tok@sx\endcsname{\def\PY@tc##1{\textcolor[rgb]{0.00,0.50,0.00}{##1}}}
\expandafter\def\csname PY@tok@m\endcsname{\def\PY@tc##1{\textcolor[rgb]{0.40,0.40,0.40}{##1}}}
\expandafter\def\csname PY@tok@gh\endcsname{\let\PY@bf=\textbf\def\PY@tc##1{\textcolor[rgb]{0.00,0.00,0.50}{##1}}}
\expandafter\def\csname PY@tok@gu\endcsname{\let\PY@bf=\textbf\def\PY@tc##1{\textcolor[rgb]{0.50,0.00,0.50}{##1}}}
\expandafter\def\csname PY@tok@gd\endcsname{\def\PY@tc##1{\textcolor[rgb]{0.63,0.00,0.00}{##1}}}
\expandafter\def\csname PY@tok@gi\endcsname{\def\PY@tc##1{\textcolor[rgb]{0.00,0.63,0.00}{##1}}}
\expandafter\def\csname PY@tok@gr\endcsname{\def\PY@tc##1{\textcolor[rgb]{1.00,0.00,0.00}{##1}}}
\expandafter\def\csname PY@tok@ge\endcsname{\let\PY@it=\textit}
\expandafter\def\csname PY@tok@gs\endcsname{\let\PY@bf=\textbf}
\expandafter\def\csname PY@tok@gp\endcsname{\let\PY@bf=\textbf\def\PY@tc##1{\textcolor[rgb]{0.00,0.00,0.50}{##1}}}
\expandafter\def\csname PY@tok@go\endcsname{\def\PY@tc##1{\textcolor[rgb]{0.53,0.53,0.53}{##1}}}
\expandafter\def\csname PY@tok@gt\endcsname{\def\PY@tc##1{\textcolor[rgb]{0.00,0.27,0.87}{##1}}}
\expandafter\def\csname PY@tok@err\endcsname{\def\PY@bc##1{\setlength{\fboxsep}{0pt}\fcolorbox[rgb]{1.00,0.00,0.00}{1,1,1}{\strut ##1}}}
\expandafter\def\csname PY@tok@kc\endcsname{\let\PY@bf=\textbf\def\PY@tc##1{\textcolor[rgb]{0.00,0.50,0.00}{##1}}}
\expandafter\def\csname PY@tok@kd\endcsname{\let\PY@bf=\textbf\def\PY@tc##1{\textcolor[rgb]{0.00,0.50,0.00}{##1}}}
\expandafter\def\csname PY@tok@kn\endcsname{\let\PY@bf=\textbf\def\PY@tc##1{\textcolor[rgb]{0.00,0.50,0.00}{##1}}}
\expandafter\def\csname PY@tok@kr\endcsname{\let\PY@bf=\textbf\def\PY@tc##1{\textcolor[rgb]{0.00,0.50,0.00}{##1}}}
\expandafter\def\csname PY@tok@bp\endcsname{\def\PY@tc##1{\textcolor[rgb]{0.00,0.50,0.00}{##1}}}
\expandafter\def\csname PY@tok@fm\endcsname{\def\PY@tc##1{\textcolor[rgb]{0.00,0.00,1.00}{##1}}}
\expandafter\def\csname PY@tok@vc\endcsname{\def\PY@tc##1{\textcolor[rgb]{0.10,0.09,0.49}{##1}}}
\expandafter\def\csname PY@tok@vg\endcsname{\def\PY@tc##1{\textcolor[rgb]{0.10,0.09,0.49}{##1}}}
\expandafter\def\csname PY@tok@vi\endcsname{\def\PY@tc##1{\textcolor[rgb]{0.10,0.09,0.49}{##1}}}
\expandafter\def\csname PY@tok@vm\endcsname{\def\PY@tc##1{\textcolor[rgb]{0.10,0.09,0.49}{##1}}}
\expandafter\def\csname PY@tok@sa\endcsname{\def\PY@tc##1{\textcolor[rgb]{0.73,0.13,0.13}{##1}}}
\expandafter\def\csname PY@tok@sb\endcsname{\def\PY@tc##1{\textcolor[rgb]{0.73,0.13,0.13}{##1}}}
\expandafter\def\csname PY@tok@sc\endcsname{\def\PY@tc##1{\textcolor[rgb]{0.73,0.13,0.13}{##1}}}
\expandafter\def\csname PY@tok@dl\endcsname{\def\PY@tc##1{\textcolor[rgb]{0.73,0.13,0.13}{##1}}}
\expandafter\def\csname PY@tok@s2\endcsname{\def\PY@tc##1{\textcolor[rgb]{0.73,0.13,0.13}{##1}}}
\expandafter\def\csname PY@tok@sh\endcsname{\def\PY@tc##1{\textcolor[rgb]{0.73,0.13,0.13}{##1}}}
\expandafter\def\csname PY@tok@s1\endcsname{\def\PY@tc##1{\textcolor[rgb]{0.73,0.13,0.13}{##1}}}
\expandafter\def\csname PY@tok@mb\endcsname{\def\PY@tc##1{\textcolor[rgb]{0.40,0.40,0.40}{##1}}}
\expandafter\def\csname PY@tok@mf\endcsname{\def\PY@tc##1{\textcolor[rgb]{0.40,0.40,0.40}{##1}}}
\expandafter\def\csname PY@tok@mh\endcsname{\def\PY@tc##1{\textcolor[rgb]{0.40,0.40,0.40}{##1}}}
\expandafter\def\csname PY@tok@mi\endcsname{\def\PY@tc##1{\textcolor[rgb]{0.40,0.40,0.40}{##1}}}
\expandafter\def\csname PY@tok@il\endcsname{\def\PY@tc##1{\textcolor[rgb]{0.40,0.40,0.40}{##1}}}
\expandafter\def\csname PY@tok@mo\endcsname{\def\PY@tc##1{\textcolor[rgb]{0.40,0.40,0.40}{##1}}}
\expandafter\def\csname PY@tok@ch\endcsname{\let\PY@it=\textit\def\PY@tc##1{\textcolor[rgb]{0.25,0.50,0.50}{##1}}}
\expandafter\def\csname PY@tok@cm\endcsname{\let\PY@it=\textit\def\PY@tc##1{\textcolor[rgb]{0.25,0.50,0.50}{##1}}}
\expandafter\def\csname PY@tok@cpf\endcsname{\let\PY@it=\textit\def\PY@tc##1{\textcolor[rgb]{0.25,0.50,0.50}{##1}}}
\expandafter\def\csname PY@tok@c1\endcsname{\let\PY@it=\textit\def\PY@tc##1{\textcolor[rgb]{0.25,0.50,0.50}{##1}}}
\expandafter\def\csname PY@tok@cs\endcsname{\let\PY@it=\textit\def\PY@tc##1{\textcolor[rgb]{0.25,0.50,0.50}{##1}}}

\def\PYZbs{\char`\\}
\def\PYZus{\char`\_}
\def\PYZob{\char`\{}
\def\PYZcb{\char`\}}
\def\PYZca{\char`\^}
\def\PYZam{\char`\&}
\def\PYZlt{\char`\<}
\def\PYZgt{\char`\>}
\def\PYZsh{\char`\#}
\def\PYZpc{\char`\%}
\def\PYZdl{\char`\$}
\def\PYZhy{\char`\-}
\def\PYZsq{\char`\'}
\def\PYZdq{\char`\"}
\def\PYZti{\char`\~}
% for compatibility with earlier versions
\def\PYZat{@}
\def\PYZlb{[}
\def\PYZrb{]}
\makeatother


    % Exact colors from NB
    \definecolor{incolor}{rgb}{0.0, 0.0, 0.5}
    \definecolor{outcolor}{rgb}{0.545, 0.0, 0.0}



    
    % Prevent overflowing lines due to hard-to-break entities
    \sloppy 
    % Setup hyperref package
    \hypersetup{
      breaklinks=true,  % so long urls are correctly broken across lines
      colorlinks=true,
      urlcolor=urlcolor,
      linkcolor=linkcolor,
      citecolor=citecolor,
      }
    % Slightly bigger margins than the latex defaults
    
    \geometry{verbose,tmargin=1in,bmargin=1in,lmargin=1in,rmargin=1in}
    
    

    \begin{document}
    
    
    \maketitle
    
    

    
    \begin{Verbatim}[commandchars=\\\{\}]
{\color{incolor}In [{\color{incolor}1}]:} \PY{k+kn}{import} \PY{n+nn}{numpy} \PY{k}{as} \PY{n+nn}{np}
        \PY{k+kn}{import} \PY{n+nn}{pandas} \PY{k}{as} \PY{n+nn}{pd}
        \PY{k+kn}{import} \PY{n+nn}{scipy}\PY{n+nn}{.}\PY{n+nn}{stats}
\end{Verbatim}


    \section{Q3}\label{q3}

\subsubsection{True value}\label{true-value}

It is easy to compute integral term \[
\theta = \int_{0}^{1} \frac{e^x-1}{e-1} dx = \frac{e-2}{e-1}
\] Thus true value of \(\theta\) is \((e-2)/(e-1)\)

    Fix the number of iteration \(n=1000\)

    \begin{Verbatim}[commandchars=\\\{\}]
{\color{incolor}In [{\color{incolor}2}]:} \PY{n}{n}\PY{o}{=}\PY{l+m+mi}{1000}
\end{Verbatim}


    \subsection{(a) Cude Monte Carlo}\label{a-cude-monte-carlo}

Let \(X = \{x_1, x_2 , \dots , x_n\}\) and \(x_i \sim^{iid} Unif(0,1)\)
for \(i=1,\dots,n\)\\
Suppose \(h(x) = \frac{e^x-1}{e-1}\) and \(f(x) =1\) \[
\begin{align*}
E_f[h(X)] =  \int_{0}^{1} \frac{e^x-1}{e-1} dx &\approx \frac{1}{n} \sum_{i=1}^{n}h(x_i)\\
\hat{\theta}^{crude}&=\frac{1}{n}\sum_{i=1}^{n}\frac{e^x_i-1}{e-1}
\end{align*}
\]

    \subsubsection{bias}\label{bias}

\[
bias_\theta (\hat{\theta}^{crude})= E[\hat{\theta}^{crude}] - \theta
\] First calculate the expectation of \(\frac{e^x_i-1}{e-1}\). As
\(x_i\)'s follow \(Unif(0,1)\) \[
E_f\left[\frac{e^x_i-1}{e-1}\right] = \int_{0}^{1}\frac{e^x_i-1}{e-1}\cdot 1 dx_i = \frac{e-2}{e-1}
\] Since, \(x_i\)'s are independent \[
E[\hat{\theta}^{crude}] = \frac{1}{n}\sum_{i=1}^{n} E\left[\frac{e^x_i-1}{e-1}\right] = \frac{e-2}{e-1} = \theta
\] Then bias of Cude Monte Carlo Estimator is \[
bias_\theta (\hat{\theta}^{crude})= E[\hat{\theta}^{crude}] - \theta = 0
\]

    \subsubsection{variance}\label{variance}

\[
\begin{align*}
Var[\hat{\theta}^{crude}]  &= Var\left[\frac{1}{n}\sum_{i=1}^{n}\frac{e^x_i-1}{e-1}\right]\\
& = \frac{1}{n^2 (e-1)^2}\sum_{i=1}^{n}Var[e^x_i-1]\\
& = \frac{1}{n^2 (e-1)^2}\sum_{i=1}^{n}Var[e^x_i]\\
& = \frac{1}{n^2 (e-1)^2}\sum_{i=1}^{n}\left[ E[e^{2\cdot x_i}] - E[e^x_i]^2 \right]
\end{align*}
\] as \(x_i \sim Unif(0,1)\) we can use mgf of Unifrom distribution \[
E_f[e^{tx}] = \frac{e^t -1}{t}
\] Thus \[
\begin{align*}
Var_f[\hat{\theta}^{crude}] & = \frac{1}{n^2 (e-1)^2}\sum_{i=1}^{n}\left[ \frac{e^2-1}{2} - (e-1)^2 \right]\\
&=\frac{1}{n (e-1)^2}\left[ \frac{e^2-1}{2} - (e-1)^2 \right]\\
&=\frac{3-e}{2n(e-1)}
\end{align*}
\]

    \subsubsection{MSE}\label{mse}

\(mse = bias^2 + var\). Thus, \[
MSE(\hat{\theta}^{crude}) = 0 + \frac{3-e}{2n(e-1)} = \frac{3-e}{2n(e-1)}
\]

    \begin{Verbatim}[commandchars=\\\{\}]
{\color{incolor}In [{\color{incolor}3}]:} \PY{k}{def} \PY{n+nf}{CrudeMC}\PY{p}{(}\PY{n}{n}\PY{o}{=}\PY{l+m+mi}{100}\PY{p}{,}\PY{n}{prt} \PY{o}{=} \PY{k+kc}{False}\PY{p}{)}\PY{p}{:}
            \PY{n}{theta} \PY{o}{=} \PY{p}{(}\PY{n}{np}\PY{o}{.}\PY{n}{exp}\PY{p}{(}\PY{l+m+mi}{1}\PY{p}{)}\PY{o}{\PYZhy{}}\PY{l+m+mi}{2}\PY{p}{)}\PY{o}{/}\PY{p}{(}\PY{n}{np}\PY{o}{.}\PY{n}{exp}\PY{p}{(}\PY{l+m+mi}{1}\PY{p}{)}\PY{o}{\PYZhy{}}\PY{l+m+mi}{1}\PY{p}{)}
            \PY{n}{X} \PY{o}{=} \PY{n}{np}\PY{o}{.}\PY{n}{random}\PY{o}{.}\PY{n}{uniform}\PY{p}{(}\PY{l+m+mi}{0}\PY{p}{,}\PY{l+m+mi}{1}\PY{p}{,}\PY{n}{n}\PY{p}{)}
            \PY{n}{theta\PYZus{}hat} \PY{o}{=} \PY{p}{(}\PY{p}{(}\PY{n}{np}\PY{o}{.}\PY{n}{exp}\PY{p}{(}\PY{n}{X}\PY{p}{)}\PY{o}{\PYZhy{}}\PY{l+m+mi}{1}\PY{p}{)}\PY{o}{/}\PY{p}{(}\PY{n}{np}\PY{o}{.}\PY{n}{exp}\PY{p}{(}\PY{l+m+mi}{1}\PY{p}{)}\PY{o}{\PYZhy{}}\PY{l+m+mi}{1}\PY{p}{)}\PY{p}{)}\PY{o}{.}\PY{n}{mean}\PY{p}{(}\PY{p}{)}
            \PY{n}{mse} \PY{o}{=} \PY{p}{(}\PY{l+m+mi}{3}\PY{o}{\PYZhy{}}\PY{n}{np}\PY{o}{.}\PY{n}{exp}\PY{p}{(}\PY{l+m+mi}{1}\PY{p}{)}\PY{p}{)}\PY{o}{/}\PY{p}{(}\PY{l+m+mi}{2}\PY{o}{*}\PY{n}{n}\PY{o}{*}\PY{p}{(}\PY{n}{np}\PY{o}{.}\PY{n}{exp}\PY{p}{(}\PY{l+m+mi}{1}\PY{p}{)}\PY{o}{\PYZhy{}}\PY{l+m+mi}{1}\PY{p}{)}\PY{p}{)}
            \PY{k}{if} \PY{n}{prt}\PY{o}{==}\PY{k+kc}{True}\PY{p}{:}
                \PY{n+nb}{print}\PY{p}{(}\PY{l+s+s1}{\PYZsq{}}\PY{l+s+s1}{theta estimator is : }\PY{l+s+si}{\PYZpc{}f}\PY{l+s+s1}{\PYZsq{}} \PY{o}{\PYZpc{}}\PY{k}{theta\PYZus{}hat})
                \PY{n+nb}{print}\PY{p}{(}\PY{l+s+s1}{\PYZsq{}}\PY{l+s+s1}{MSE is : }\PY{l+s+si}{\PYZpc{}f}\PY{l+s+s1}{\PYZsq{}} \PY{o}{\PYZpc{}}\PY{k}{mse})
            \PY{k}{return}\PY{p}{(}\PY{n}{theta\PYZus{}hat}\PY{p}{,}\PY{n}{mse}\PY{p}{)}
\end{Verbatim}


    \begin{Verbatim}[commandchars=\\\{\}]
{\color{incolor}In [{\color{incolor}4}]:} \PY{n}{crude\PYZus{}est}\PY{p}{,}\PY{n}{crude\PYZus{}mse} \PY{o}{=} \PY{n}{CrudeMC}\PY{p}{(}\PY{n}{n}\PY{p}{,}\PY{l+m+mi}{1}\PY{p}{)}
\end{Verbatim}


    \begin{Verbatim}[commandchars=\\\{\}]
theta estimator is : 0.415960
MSE is : 0.000082

    \end{Verbatim}

    \subsection{(b) Importance sampling}\label{b-importance-sampling}

Let \(g(x)=\frac{e^{-t}}{1-e^{-1}}\). We need to sample from \(g(x)\)
which follows truncated exponential distribution. We can use Inverse CDF
method\\
Suppose \(G(x)\) is CDF of \(g(x)\) then, \[
\begin{align*}
G(x) &= \int_{0}^{x}\frac{e^{-t}}{1-e^{-1}}dt = \frac{1 -e^{-x}}{1 -e^{-1}} \sim Unif(0,1)\\
G^{-1}(U) &= -log \left(1-U(1-e^{-1})\right)
\end{align*}
\] When \(U\sim Unif(0,1)\), \(G^{-1}(U)\) follows truncated exponentail
distribution\\
Like Crude Monte Carlo method suppose \(h(x) = \frac{e^x-1}{e-1}\) and
\(f(x) =1\) and \(x_i\)'s follows \(g(x)\) which is truncated
exponential distribution

    \[
\begin{align*}
\hat{\theta}^{Importance} = E_{g}\left[h(x)\frac{f(x)}{g(x)}\right] &\approx \frac{1}{n} \sum_{i=1}^{n}h(x_i)\frac{f(x_i)}{g(x_i)}\\
&=\frac{1}{n} \sum_{i=1}^{n}\frac{e^{x_i}-1}{e^{-(x_i-1)}}
\end{align*}
\]

    \subsubsection{bias}\label{bias}

\[
bias_\theta (\hat{\theta}^{Importance})= E[\hat{\theta}^{Importance}] - \theta
\] First calculate the expectation of
\(\frac{e^{x_i}-1}{e^{-(x_i-1)}}\)\\
As \(x_i\)'s are iid sampled from pdf \(g(x)\) \[
E_g\left[\frac{e^{x_i}-1}{e^{-(x_i-1)}}\right] =\int_{0}^{1}\frac{e^{x_i}-1}{e^{-(x_i-1)}}g(x_i)dx_i= \int_{0}^{1}\frac{e^x_i-1}{e-1}dx_i = \frac{e-2}{e-1}
\] Since, \(x_i\)'s are independent \[
E[\hat{\theta}^{Importance}] = \frac{1}{n}\sum_{i=1}^{n} E_g\left[\frac{e^{x_i}-1}{e^{-(x_i-1)}}\right] = \frac{e-2}{e-1} =\theta
\] Then bias of Importance sampling is \[
bias_\theta (\hat{\theta}^{Importance})= E[\hat{\theta}^{Importance}] - \theta = 0
\]

    \subsubsection{variance}\label{variance}

\[
\begin{align*}
Var[\hat{\theta}^{Importance}]  &= Var\left[\sum_{i=1}^{n}\frac{e^{x_i}-1}{e^{-(x_i-1)}}\right]\\
& = \frac{1}{n^2 e^2}\sum_{i=1}^{n}Var\left[\frac{e^{x_i}-1}{e^{-x_i}}\right]\\
& = \frac{1}{n^2 e^2}\sum_{i=1}^{n}Var[e^{2x_i}-e^{x_i}]\\
& = \frac{1}{n^2 e^2}\sum_{i=1}^{n}\left[ E_g[e^{4x_i}] - E_g[e^{2x_i}]^2 + E_g[e^{2x_i}] -E_g[e^{x_i}]^2 -2\left(E_g[e^{3x_i}] - E_g[e^{2x_i}]E_g[e^{x_i}]\right) \right]
\end{align*}
\] First calulate the MGF of g(x) \[
M_x(t)=E_g[e^{tx}] = \int_{0}^{1}e^{tx}\frac{e^{-x}}{1-e^{-1}}dx = \frac{e^t-e}{(t-1)(e-1)}
\] When \(t>1\) and \(M_x(1) = e/(e-1)\) Thus \[
\begin{align*}
Var_g[\hat{\theta}^{Importance}] & = \frac{1}{n e^2}\left[ M_x(4) - M_x(2)^2 + M_x(2) - M_x(1)^2 -2 \left( M_x(3) - M_x(2)M_x(1)\right)\right]\\
\end{align*}
\]

    \subsubsection{MSE}\label{mse}

\(mse = bias^2 + var\). Thus, \[
\begin{align*}
MSE(\hat{\theta}^{Importance}) &= 0 +\frac{1}{n e^2}\left[ M_x(4) - M_x(2)^2 + M_x(2) - M_x(1)^2 -2 \left( M_x(3) - M_x(2)M_x(1)\right)\right]\\ &= \frac{1}{n e^2}\left[ M_x(4) - M_x(2)^2 + M_x(2) - M_x(1)^2 -2 \left( M_x(3) - M_x(2)M_x(1)\right)\right]
\end{align*}
\]

    \begin{Verbatim}[commandchars=\\\{\}]
{\color{incolor}In [{\color{incolor}5}]:} \PY{k}{def} \PY{n+nf}{Samplegx}\PY{p}{(}\PY{n}{n}\PY{o}{=}\PY{l+m+mi}{100}\PY{p}{)}\PY{p}{:}
            \PY{n}{U} \PY{o}{=} \PY{n}{np}\PY{o}{.}\PY{n}{random}\PY{o}{.}\PY{n}{uniform}\PY{p}{(}\PY{l+m+mi}{0}\PY{p}{,}\PY{l+m+mi}{1}\PY{p}{,}\PY{n}{n}\PY{p}{)}
            \PY{n}{X} \PY{o}{=} \PY{o}{\PYZhy{}}\PY{n}{np}\PY{o}{.}\PY{n}{log}\PY{p}{(}\PY{l+m+mi}{1}\PY{o}{\PYZhy{}}\PY{n}{U}\PY{o}{*}\PY{p}{(}\PY{l+m+mi}{1}\PY{o}{\PYZhy{}}\PY{n}{np}\PY{o}{.}\PY{n}{exp}\PY{p}{(}\PY{o}{\PYZhy{}}\PY{l+m+mi}{1}\PY{p}{)}\PY{p}{)}\PY{p}{)}
            \PY{k}{return}\PY{p}{(}\PY{n}{X}\PY{p}{)}
\end{Verbatim}


    \begin{Verbatim}[commandchars=\\\{\}]
{\color{incolor}In [{\color{incolor}6}]:} \PY{k}{def} \PY{n+nf}{mgfgx}\PY{p}{(}\PY{n}{t}\PY{p}{)}\PY{p}{:}
            \PY{k}{if} \PY{n}{t}\PY{o}{==}\PY{l+m+mi}{1}\PY{p}{:}
                \PY{n}{out} \PY{o}{=} \PY{n}{np}\PY{o}{.}\PY{n}{exp}\PY{p}{(}\PY{l+m+mi}{1}\PY{p}{)}\PY{o}{/}\PY{p}{(}\PY{n}{np}\PY{o}{.}\PY{n}{exp}\PY{p}{(}\PY{l+m+mi}{1}\PY{p}{)}\PY{o}{\PYZhy{}}\PY{l+m+mi}{1}\PY{p}{)}
            \PY{k}{elif} \PY{n}{t}\PY{o}{\PYZgt{}}\PY{l+m+mi}{1}\PY{p}{:}
                \PY{n}{out} \PY{o}{=} \PY{p}{(}\PY{n}{np}\PY{o}{.}\PY{n}{exp}\PY{p}{(}\PY{n}{t}\PY{p}{)}\PY{o}{\PYZhy{}}\PY{n}{np}\PY{o}{.}\PY{n}{exp}\PY{p}{(}\PY{l+m+mi}{1}\PY{p}{)}\PY{p}{)}\PY{o}{/}\PY{p}{(}\PY{p}{(}\PY{n}{t}\PY{o}{\PYZhy{}}\PY{l+m+mi}{1}\PY{p}{)}\PY{o}{*}\PY{p}{(}\PY{n}{np}\PY{o}{.}\PY{n}{exp}\PY{p}{(}\PY{l+m+mi}{1}\PY{p}{)}\PY{o}{\PYZhy{}}\PY{l+m+mi}{1}\PY{p}{)}\PY{p}{)}
            \PY{k}{return}\PY{p}{(}\PY{n}{out}\PY{p}{)}
\end{Verbatim}


    \begin{Verbatim}[commandchars=\\\{\}]
{\color{incolor}In [{\color{incolor}7}]:} \PY{k}{def} \PY{n+nf}{importance}\PY{p}{(}\PY{n}{n}\PY{o}{=}\PY{l+m+mi}{100}\PY{p}{,}\PY{n}{prt} \PY{o}{=} \PY{k+kc}{False}\PY{p}{)}\PY{p}{:}
            \PY{n}{X} \PY{o}{=} \PY{n}{Samplegx}\PY{p}{(}\PY{n}{n}\PY{p}{)}
            \PY{n}{theta\PYZus{}hat} \PY{o}{=} \PY{p}{(}\PY{p}{(}\PY{n}{np}\PY{o}{.}\PY{n}{exp}\PY{p}{(}\PY{n}{X}\PY{p}{)}\PY{o}{\PYZhy{}}\PY{l+m+mi}{1}\PY{p}{)}\PY{o}{/}\PY{p}{(}\PY{n}{np}\PY{o}{.}\PY{n}{exp}\PY{p}{(}\PY{o}{\PYZhy{}}\PY{n}{X}\PY{o}{+}\PY{l+m+mi}{1}\PY{p}{)}\PY{p}{)}\PY{p}{)}\PY{o}{.}\PY{n}{mean}\PY{p}{(}\PY{p}{)}
            \PY{n}{mse} \PY{o}{=} \PY{p}{(}\PY{n}{mgfgx}\PY{p}{(}\PY{l+m+mi}{4}\PY{p}{)} \PY{o}{\PYZhy{}} \PY{n}{mgfgx}\PY{p}{(}\PY{l+m+mi}{2}\PY{p}{)}\PY{o}{*}\PY{o}{*}\PY{l+m+mi}{2} \PY{o}{+} \PY{n}{mgfgx}\PY{p}{(}\PY{l+m+mi}{2}\PY{p}{)} \PY{o}{\PYZhy{}} \PY{n}{mgfgx}\PY{p}{(}\PY{l+m+mi}{1}\PY{p}{)}\PY{o}{*}\PY{o}{*}\PY{l+m+mi}{2} \PY{o}{\PYZhy{}} \PY{l+m+mi}{2}\PY{o}{*}\PY{p}{(}\PY{n}{mgfgx}\PY{p}{(}\PY{l+m+mi}{3}\PY{p}{)}\PY{o}{\PYZhy{}}\PY{n}{mgfgx}\PY{p}{(}\PY{l+m+mi}{2}\PY{p}{)}\PY{o}{*}\PY{n}{mgfgx}\PY{p}{(}\PY{l+m+mi}{1}\PY{p}{)}\PY{p}{)}\PY{p}{)}\PY{o}{/}\PY{p}{(}\PY{n}{n}\PY{o}{*}\PY{p}{(}\PY{n}{np}\PY{o}{.}\PY{n}{exp}\PY{p}{(}\PY{l+m+mi}{2}\PY{p}{)}\PY{p}{)}\PY{p}{)}
            \PY{k}{if} \PY{n}{prt}\PY{o}{==}\PY{k+kc}{True}\PY{p}{:}
                \PY{n+nb}{print}\PY{p}{(}\PY{l+s+s1}{\PYZsq{}}\PY{l+s+s1}{theta estimator is : }\PY{l+s+si}{\PYZpc{}f}\PY{l+s+s1}{\PYZsq{}} \PY{o}{\PYZpc{}}\PY{k}{theta\PYZus{}hat})
                \PY{n+nb}{print}\PY{p}{(}\PY{l+s+s1}{\PYZsq{}}\PY{l+s+s1}{MSE is : }\PY{l+s+si}{\PYZpc{}f}\PY{l+s+s1}{\PYZsq{}} \PY{o}{\PYZpc{}}\PY{k}{mse})
            \PY{k}{return}\PY{p}{(}\PY{n}{theta\PYZus{}hat}\PY{p}{,}\PY{n}{mse}\PY{p}{)}
\end{Verbatim}


    \begin{Verbatim}[commandchars=\\\{\}]
{\color{incolor}In [{\color{incolor}8}]:} \PY{n}{imps\PYZus{}est}\PY{p}{,}\PY{n}{imps\PYZus{}mse} \PY{o}{=} \PY{n}{importance}\PY{p}{(}\PY{n}{n}\PY{p}{)}
\end{Verbatim}


    \subsection{(c) Control Variates}\label{c-control-variates}

    as (a) Suppose \(h(x) = \frac{e^x-1}{e-1}\), \(f(x) =1\) and
\(g(x)=\frac{x}{e-1}\) \[
\begin{align*}
 E_f[h(X)]&=E_f[g(X)] + E_f[h(X)-g(X)]\\
\hat{\theta}^{CV} &= E_f[g(X)] +\frac{1}{n}\sum_{i=1}^{n}\left( h(x_i) -g(x_i) \right)\\
&=\frac{1}{2(e-1)} + \frac{1}{n(e-1)}\sum_{i=1}^{n} (e^{x_i}-x_i -1)
\end{align*}
\]

    \subsubsection{bias}\label{bias}

\[
E_f[x] = 0.5
\] From (a) \[
E_f[e^{tx}] = \frac{e^t -1}{t}
\] Thus \[
E\left[\hat{\theta}^{CV}\right] = \frac{1}{2(e-1)} + \frac{1}{n(e-1)}\sum_{i=1}^{n}\left( E\left[e^{x_i}\right] - E\left[x_i\right] -1\right) = \frac{e-2}{e-1} = \theta
\] Then bias of Control variates is \[
bias_\theta (\hat{\theta}^{CV})= E[\hat{\theta}^{CV}] - \theta = 0
\]

    \subsubsection{variance}\label{variance}

as \(x_i\)'s follows \(Unif(0,1)\) \[
\begin{align*}
E[x_i^2] &= var[x_i] + E[x_i]^2 = \frac{1}{3}\\
E[e^{x_i}x_i] &= \int_{0}^{1} e^{x_i}x_i dx_i = 1
\end{align*}
\] Variance of Control variates estimator is \[
\begin{align*}
Var_f[\hat{\theta}^{CV}] & = \frac{1}{n^2 (e-1)^2}\sum_{i=1}^{n}Var_f\left[e^{x_i} - x_i -1\right]\\
&=\frac{1}{n^2 (e-1)^2}\sum_{i=1}^{n}Var_f\left[e^{x_i} - x_i\right]\\
&=\frac{1}{n^2 (e-1)^2}\sum_{i=1}^{n}Var_f\left[e^{x_i} - x_i\right]\\
&=\frac{1}{n^2 (e-1)^2}\sum_{i=1}^{n}\left[E_f[e^{2x_i}]-E_f[e^{x_i}]^2 + E_f[x_i^2] - E_f[x_i]^2 - 2\left(E[e^{x_i}x_i] - E_f[e^{x_i}]E_f[x_i]\right)\right]\\
&=\frac{1}{n (e-1)^2}\left[ (e-1)\left( \frac{5-e}{2}\right) - \frac{23}{12} \right]
\end{align*}
\]

    \subsubsection{MSE}\label{mse}

\(mse = bias^2 + var\). Thus, \[
\begin{align*}
MSE(\hat{\theta}^{CV}) &= 0 + \frac{1}{n (e-1)^2}\left[ (e-1)\left( \frac{5-e}{2}\right) - \frac{23}{12} \right]\\ 
&= \frac{1}{n (e-1)^2}\left[ (e-1)\left( \frac{5-e}{2}\right) - \frac{23}{12} \right]
\end{align*}
\]

    \begin{Verbatim}[commandchars=\\\{\}]
{\color{incolor}In [{\color{incolor}9}]:} \PY{k}{def} \PY{n+nf}{ControlVariates}\PY{p}{(}\PY{n}{n}\PY{o}{=}\PY{l+m+mi}{100}\PY{p}{,} \PY{n}{prt} \PY{o}{=} \PY{k+kc}{False}\PY{p}{)}\PY{p}{:}
            \PY{n}{theta} \PY{o}{=} \PY{p}{(}\PY{n}{np}\PY{o}{.}\PY{n}{exp}\PY{p}{(}\PY{l+m+mi}{1}\PY{p}{)}\PY{o}{\PYZhy{}}\PY{l+m+mi}{2}\PY{p}{)}\PY{o}{/}\PY{p}{(}\PY{n}{np}\PY{o}{.}\PY{n}{exp}\PY{p}{(}\PY{l+m+mi}{1}\PY{p}{)}\PY{o}{\PYZhy{}}\PY{l+m+mi}{1}\PY{p}{)}
            \PY{n}{X} \PY{o}{=} \PY{n}{np}\PY{o}{.}\PY{n}{random}\PY{o}{.}\PY{n}{uniform}\PY{p}{(}\PY{l+m+mi}{0}\PY{p}{,}\PY{l+m+mi}{1}\PY{p}{,}\PY{n}{n}\PY{p}{)}
            \PY{n}{theta\PYZus{}hat} \PY{o}{=} \PY{l+m+mi}{1}\PY{o}{/}\PY{p}{(}\PY{l+m+mi}{2}\PY{o}{*}\PY{p}{(}\PY{n}{np}\PY{o}{.}\PY{n}{exp}\PY{p}{(}\PY{l+m+mi}{1}\PY{p}{)}\PY{o}{\PYZhy{}}\PY{l+m+mi}{1}\PY{p}{)}\PY{p}{)} \PY{o}{+} \PY{p}{(}\PY{n}{np}\PY{o}{.}\PY{n}{exp}\PY{p}{(}\PY{n}{X}\PY{p}{)} \PY{o}{\PYZhy{}} \PY{n}{X} \PY{o}{\PYZhy{}}\PY{l+m+mi}{1}\PY{p}{)}\PY{o}{.}\PY{n}{mean}\PY{p}{(}\PY{p}{)}\PY{o}{/}\PY{p}{(}\PY{n}{np}\PY{o}{.}\PY{n}{exp}\PY{p}{(}\PY{l+m+mi}{1}\PY{p}{)}\PY{o}{\PYZhy{}}\PY{l+m+mi}{1}\PY{p}{)}
            \PY{n}{mse} \PY{o}{=} \PY{p}{(}\PY{p}{(}\PY{n}{np}\PY{o}{.}\PY{n}{exp}\PY{p}{(}\PY{l+m+mi}{1}\PY{p}{)}\PY{o}{\PYZhy{}}\PY{l+m+mi}{1}\PY{p}{)}\PY{o}{*}\PY{p}{(}\PY{p}{(}\PY{l+m+mi}{5}\PY{o}{\PYZhy{}}\PY{n}{np}\PY{o}{.}\PY{n}{exp}\PY{p}{(}\PY{l+m+mi}{1}\PY{p}{)}\PY{p}{)}\PY{o}{/}\PY{l+m+mi}{2}\PY{p}{)} \PY{o}{\PYZhy{}} \PY{l+m+mi}{23}\PY{o}{/}\PY{l+m+mi}{12}\PY{p}{)}\PY{o}{/}\PY{p}{(}\PY{n}{n}\PY{o}{*}\PY{p}{(}\PY{n}{np}\PY{o}{.}\PY{n}{exp}\PY{p}{(}\PY{l+m+mi}{1}\PY{p}{)}\PY{o}{\PYZhy{}}\PY{l+m+mi}{1}\PY{p}{)}\PY{o}{*}\PY{o}{*}\PY{l+m+mi}{2}\PY{p}{)}
            \PY{k}{if} \PY{n}{prt}\PY{o}{==}\PY{k+kc}{True}\PY{p}{:}
                \PY{n+nb}{print}\PY{p}{(}\PY{l+s+s1}{\PYZsq{}}\PY{l+s+s1}{theta estimator is : }\PY{l+s+si}{\PYZpc{}f}\PY{l+s+s1}{\PYZsq{}} \PY{o}{\PYZpc{}}\PY{k}{theta\PYZus{}hat})
                \PY{n+nb}{print}\PY{p}{(}\PY{l+s+s1}{\PYZsq{}}\PY{l+s+s1}{MSE is : }\PY{l+s+si}{\PYZpc{}f}\PY{l+s+s1}{\PYZsq{}} \PY{o}{\PYZpc{}}\PY{k}{mse})
            \PY{k}{return}\PY{p}{(}\PY{n}{theta\PYZus{}hat}\PY{p}{,}\PY{n}{mse}\PY{p}{)}
\end{Verbatim}


    \begin{Verbatim}[commandchars=\\\{\}]
{\color{incolor}In [{\color{incolor}10}]:} \PY{n}{CV\PYZus{}est} \PY{p}{,} \PY{n}{CV\PYZus{}mse} \PY{o}{=} \PY{n}{ControlVariates}\PY{p}{(}\PY{n}{n}\PY{p}{,}\PY{l+m+mi}{1}\PY{p}{)}
\end{Verbatim}


    \begin{Verbatim}[commandchars=\\\{\}]
theta estimator is : 0.423796
MSE is : 0.000015

    \end{Verbatim}

    \subsection{Antithetic Variates}\label{antithetic-variates}

As \(x\sim Unif(0,1)\), \(1-x \sim Unif(0,1)\). Thus \[
\begin{align*}
\hat{\theta}^{Antithetic} & = \frac{1}{2n}\sum_{i=1}^{n} \left[ h(x_i) - h(1-x_i) \right]\\
&= \frac{1}{2n}\sum_{i=1}^{n}\frac{e^{x_i} + e^{1-x_i} -2}{e-1}
\end{align*}
\]

    \subsubsection{bias}\label{bias}

As \(x\sim Unif(0,1)\), \(1-x \sim Unif(0,1)\). Thus \[
E[e^{x-i}] = E[e^{1-x_i}] = e-1
\] Therefore \[
\begin{align*}
E[\hat{\theta}^{Antithetic}]&= E\left[\frac{1}{2n}\sum_{i=1}^{n}\frac{e^{x_i} + e^{1-x_i} -2}{e-1}\right]\\
& = \frac{1}{2n}\sum_{i=1}^{n}\frac{E[e^{x_i}] + E[e^{1-x_i}] -2}{e-1}\\
& = \frac{e-2}{e-1} = \theta
\end{align*}
\] Then bias of Antithetic variates is \[
bias_\theta (\hat{\theta}^{Antithetic})= E[\hat{\theta}^{Antithetic}] - \theta = 0
\]

    \subsubsection{variance}\label{variance}

as \(x_i\)'s follows \(Unif(0,1)\) and \(1-x_i\)'s follows \(Unif(0,1)\)
We can use same expectation and mgf\\
Variance of Antithetic variates estimator is \[
\begin{align*}
Var_f[\hat{\theta}^{Antithetic}] & = \frac{1}{4n^2 (e-1)^2}\sum_{i=1}^{n}Var_f\left[e^{x_i} - e^{1-x_i} -2\right]\\
&= \frac{1}{4n^2 (e-1)^2}\sum_{i=1}^{n}Var_f\left[e^{x_i} - e^{1-x_i}\right]\\
&=\frac{1}{4n^2 (e-1)^2}\sum_{i=1}^{n}2\left[E_f[e^{2x_i}]-E_f[e^{x_i}]^2 -\left(E[e] - E_f[e^{x_i}]^2\right)\right]\\
&=\frac{1}{4n (e-1)^2}\left[e^2-1 -4(e-1)^2 + 2e\right]\\
&=\frac{1}{4n (e-1)^2}\left[-3e^2 + 10e -5\right]
\end{align*}
\]

    \subsubsection{MSE}\label{mse}

\(mse = bias^2 + var\). Thus, \[
\begin{align*}
MSE(\hat{\theta}^{Antithetic}) &= 0 + \frac{1}{4n (e-1)^2}\left[-3e^2 + 10e -5\right]\\ 
&= \frac{1}{4n (e-1)^2}\left[-3e^2 + 10e -5\right]
\end{align*}
\]

    \begin{Verbatim}[commandchars=\\\{\}]
{\color{incolor}In [{\color{incolor}11}]:} \PY{k}{def} \PY{n+nf}{Antithetic}\PY{p}{(}\PY{n}{n}\PY{o}{=}\PY{l+m+mi}{100}\PY{p}{,}\PY{n}{prt} \PY{o}{=} \PY{k+kc}{False}\PY{p}{)}\PY{p}{:}
             \PY{n}{theta} \PY{o}{=} \PY{p}{(}\PY{n}{np}\PY{o}{.}\PY{n}{exp}\PY{p}{(}\PY{l+m+mi}{1}\PY{p}{)}\PY{o}{\PYZhy{}}\PY{l+m+mi}{2}\PY{p}{)}\PY{o}{/}\PY{p}{(}\PY{n}{np}\PY{o}{.}\PY{n}{exp}\PY{p}{(}\PY{l+m+mi}{1}\PY{p}{)}\PY{o}{\PYZhy{}}\PY{l+m+mi}{1}\PY{p}{)}
             \PY{n}{X} \PY{o}{=} \PY{n}{np}\PY{o}{.}\PY{n}{random}\PY{o}{.}\PY{n}{uniform}\PY{p}{(}\PY{l+m+mi}{0}\PY{p}{,}\PY{l+m+mi}{1}\PY{p}{,}\PY{n}{n}\PY{p}{)}
             \PY{n}{theta\PYZus{}hat} \PY{o}{=} \PY{p}{(}\PY{p}{(}\PY{n}{np}\PY{o}{.}\PY{n}{exp}\PY{p}{(}\PY{n}{X}\PY{p}{)} \PY{o}{+} \PY{n}{np}\PY{o}{.}\PY{n}{exp}\PY{p}{(}\PY{l+m+mi}{1}\PY{o}{\PYZhy{}}\PY{n}{X}\PY{p}{)} \PY{o}{\PYZhy{}}\PY{l+m+mi}{2}\PY{p}{)}\PY{o}{/}\PY{p}{(}\PY{n}{np}\PY{o}{.}\PY{n}{exp}\PY{p}{(}\PY{l+m+mi}{1}\PY{p}{)}\PY{o}{\PYZhy{}}\PY{l+m+mi}{1}\PY{p}{)}\PY{p}{)}\PY{o}{.}\PY{n}{mean}\PY{p}{(}\PY{p}{)}\PY{o}{/}\PY{l+m+mi}{2}
             \PY{n}{mse} \PY{o}{=} \PY{p}{(}\PY{o}{\PYZhy{}}\PY{l+m+mi}{3} \PY{o}{*} \PY{n}{np}\PY{o}{.}\PY{n}{exp}\PY{p}{(}\PY{l+m+mi}{1}\PY{p}{)}\PY{o}{*}\PY{o}{*}\PY{l+m+mi}{2} \PY{o}{+} \PY{l+m+mi}{10} \PY{o}{*}\PY{n}{np}\PY{o}{.}\PY{n}{exp}\PY{p}{(}\PY{l+m+mi}{1}\PY{p}{)} \PY{o}{\PYZhy{}}\PY{l+m+mi}{5}\PY{p}{)}\PY{o}{/}\PY{p}{(}\PY{l+m+mi}{4}\PY{o}{*}\PY{n}{n}\PY{o}{*}\PY{p}{(}\PY{p}{(}\PY{n}{np}\PY{o}{.}\PY{n}{exp}\PY{p}{(}\PY{l+m+mi}{1}\PY{p}{)}\PY{o}{\PYZhy{}}\PY{l+m+mi}{1}\PY{p}{)}\PY{o}{*}\PY{o}{*}\PY{l+m+mi}{2}\PY{p}{)}\PY{p}{)}
             \PY{k}{if} \PY{n}{prt}\PY{o}{==}\PY{k+kc}{True}\PY{p}{:}
                 \PY{n+nb}{print}\PY{p}{(}\PY{l+s+s1}{\PYZsq{}}\PY{l+s+s1}{theta estimator is : }\PY{l+s+si}{\PYZpc{}f}\PY{l+s+s1}{\PYZsq{}} \PY{o}{\PYZpc{}}\PY{k}{theta\PYZus{}hat})
                 \PY{n+nb}{print}\PY{p}{(}\PY{l+s+s1}{\PYZsq{}}\PY{l+s+s1}{MSE is : }\PY{l+s+si}{\PYZpc{}f}\PY{l+s+s1}{\PYZsq{}} \PY{o}{\PYZpc{}}\PY{k}{mse})
             \PY{k}{return}\PY{p}{(}\PY{n}{theta\PYZus{}hat}\PY{p}{,}\PY{n}{mse}\PY{p}{)}
\end{Verbatim}


    \begin{Verbatim}[commandchars=\\\{\}]
{\color{incolor}In [{\color{incolor}12}]:} \PY{n}{ant\PYZus{}est}\PY{p}{,} \PY{n}{ant\PYZus{}mse} \PY{o}{=} \PY{n}{Antithetic}\PY{p}{(}\PY{n}{n}\PY{p}{,}\PY{l+m+mi}{1}\PY{p}{)}
\end{Verbatim}


    \begin{Verbatim}[commandchars=\\\{\}]
theta estimator is : 0.416557
MSE is : 0.000001

    \end{Verbatim}

    \subsection{Comparison}\label{comparison}

MSE of (a) \textasciitilde{} (c) methods can be used for comparison of
efficiency when sample size \(n\) is fixed

    \begin{Verbatim}[commandchars=\\\{\}]
{\color{incolor}In [{\color{incolor}13}]:} \PY{n}{n} \PY{o}{=} \PY{l+m+mi}{1000}
         \PY{n}{comp} \PY{o}{=} \PY{n}{pd}\PY{o}{.}\PY{n}{DataFrame}\PY{p}{(}\PY{p}{[}\PY{p}{(}\PY{p}{(}\PY{n}{np}\PY{o}{.}\PY{n}{exp}\PY{p}{(}\PY{l+m+mi}{1}\PY{p}{)}\PY{o}{\PYZhy{}}\PY{l+m+mi}{2}\PY{p}{)}\PY{o}{/}\PY{p}{(}\PY{n}{np}\PY{o}{.}\PY{n}{exp}\PY{p}{(}\PY{l+m+mi}{1}\PY{p}{)}\PY{o}{\PYZhy{}}\PY{l+m+mi}{1}\PY{p}{)}\PY{p}{,}\PY{l+m+mi}{0}\PY{p}{)}\PY{p}{,}\PY{n}{CrudeMC}\PY{p}{(}\PY{n}{n}\PY{p}{)}\PY{p}{,}\PY{n}{importance}\PY{p}{(}\PY{n}{n}\PY{p}{)}\PY{p}{,}\PY{n}{ControlVariates}\PY{p}{(}\PY{n}{n}\PY{p}{)}\PY{p}{,}\PY{n}{Antithetic}\PY{p}{(}\PY{n}{n}\PY{p}{)}\PY{p}{]}\PY{p}{)}
         \PY{n}{comp}\PY{o}{.}\PY{n}{index} \PY{o}{=} \PY{p}{[}\PY{l+s+s1}{\PYZsq{}}\PY{l+s+s1}{True Theta}\PY{l+s+s1}{\PYZsq{}}\PY{p}{,}\PY{l+s+s1}{\PYZsq{}}\PY{l+s+s1}{Crude Monte Carlo}\PY{l+s+s1}{\PYZsq{}}\PY{p}{,}\PY{l+s+s1}{\PYZsq{}}\PY{l+s+s1}{Importance sampling}\PY{l+s+s1}{\PYZsq{}}\PY{p}{,}\PY{l+s+s1}{\PYZsq{}}\PY{l+s+s1}{ Control variates}\PY{l+s+s1}{\PYZsq{}}\PY{p}{,}\PY{l+s+s1}{\PYZsq{}}\PY{l+s+s1}{Antithetic variates}\PY{l+s+s1}{\PYZsq{}}\PY{p}{]}
         \PY{n}{comp}\PY{o}{.}\PY{n}{columns} \PY{o}{=} \PY{p}{[}\PY{l+s+s1}{\PYZsq{}}\PY{l+s+s1}{Point Estimator}\PY{l+s+s1}{\PYZsq{}}\PY{p}{,}\PY{l+s+s1}{\PYZsq{}}\PY{l+s+s1}{MSE}\PY{l+s+s1}{\PYZsq{}}\PY{p}{]}
         \PY{n}{base} \PY{o}{=} \PY{n}{comp}\PY{p}{[}\PY{l+s+s1}{\PYZsq{}}\PY{l+s+s1}{MSE}\PY{l+s+s1}{\PYZsq{}}\PY{p}{]}\PY{p}{[}\PY{n}{comp}\PY{p}{[}\PY{l+s+s1}{\PYZsq{}}\PY{l+s+s1}{MSE}\PY{l+s+s1}{\PYZsq{}}\PY{p}{]}\PY{o}{.}\PY{n}{index} \PY{o}{==} \PY{l+s+s1}{\PYZsq{}}\PY{l+s+s1}{Crude Monte Carlo}\PY{l+s+s1}{\PYZsq{}}\PY{p}{]}
         \PY{n}{comp}\PY{p}{[}\PY{l+s+s1}{\PYZsq{}}\PY{l+s+s1}{Efficiency}\PY{l+s+s1}{\PYZsq{}}\PY{p}{]} \PY{o}{=} \PY{n}{comp}\PY{p}{[}\PY{l+s+s1}{\PYZsq{}}\PY{l+s+s1}{MSE}\PY{l+s+s1}{\PYZsq{}}\PY{p}{]}\PY{o}{/}\PY{n+nb}{float}\PY{p}{(}\PY{n}{base}\PY{p}{)}
         \PY{n}{display}\PY{p}{(}\PY{n}{comp}\PY{p}{)}
\end{Verbatim}


    
    \begin{verbatim}
                     Point Estimator       MSE  Efficiency
True Theta                  0.418023  0.000000    0.000000
Crude Monte Carlo           0.422746  0.000082    1.000000
Importance sampling         0.401052  0.000187    2.284921
 Control variates           0.425319  0.000015    0.180349
Antithetic variates         0.417701  0.000001    0.016165
    \end{verbatim}

    
    As all methods are unbiased estimator we can conclude the efficiency by
MSE. Antithetic variates method is most efficiency. Second is Control
variates method. Third is Crude Monte Carlo mehod. Importance sampling
shows worst efficiency

    \section{Q4}\label{q4}

Define the varibles \[
\begin{align*}
h(X) = I(X>c)\\
f(X) = \phi(X)\\
g(X) = \phi(X-b)
\end{align*}
\] where \(\phi(\cdot)\) denotes a standard normal density function \[
\begin{align*}
p &= E_f[h(Z)] = E_g \left[ h(Z)\frac{f(Z)}{g(Z)}\right]\\
&\approx \frac{1}{n} \sum_{i=1}^{n}h(z_i)\frac{f(z_i)}{g(z_i)} = \tilde{p}(b)
\end{align*} 
\]

    \subsection{\texorpdfstring{(a) find
\(b^*\)}{(a) find b\^{}*}}\label{a-find-b}

\[
\begin{align*}
Var[\tilde{p}(b)] &\propto var_g\left(h(z)\frac{f(z)}{g(z))}\right) \\
&=\frac{1}{n^2} \sum_{i=1}^{n}\left\{E_g\left[\left(h(z_i)\frac{f(z_i)}{g(z_i)}\right)^2\right] - E_g\left[h(z_i)\frac{f(z_i)}{g(z_i)}\right]^2\right\}
\end{align*}
\] When \(S(\cdot)\) be the survival function for a standard normal
distribution \[
\begin{align*}
E_g\left[\left(h(z_i)\frac{f(z_i)}{g(z_i)}\right)^2\right] &= e^{b^2}\int_{-\infty}^{\infty} I(z_i > c) \phi(z_i) dz_i = e^{b^2} S(c+b)\\
E_g\left[h(z_i)\frac{f(z_i)}{g(z_i)}\right] &= E_f[h(z_i)] = S(c)
\end{align*}
\] Thus, \[
\begin{align*}
Var_g[\tilde{p}(b)] &= \frac{1}{n^2} \sum_{i=1}^{n} \left[ e^{b^2} S(c+b) - S^2(c)\right]\\
&=\frac{1}{n}\left[ e^{b^2} S(c+b) - S^2(c)\right]\\
\end{align*}
\] \(Var_g[\tilde{p}(b)]\) minimized when
\(\partial Var_g[\tilde{p}(b)] / \partial b\) \[
\begin{align*}
\frac{\partial Var_g[\tilde{p}(b)]}{ \partial b}&=^{let}0\\
2b \cdot e^{b^2}S(b+c) - e^{b^2}\phi(b+c)&=0\\
\end{align*}
\] Therefore condition for \(b^*\) that minimize \(Var_g[\tilde{p}(b)]\)
is, \[
2b \cdot e^{b^2}S(b+c) - e^{b^2}\phi(b+c)=0\\
\rightarrow2b^*\cdot S(b^*+c) = \phi(b^*+c)
\]

    \subsection{(b) Find boundary
condition}\label{b-find-boundary-condition}

From (a) \[
\frac{S(b^*+c)}{\phi(b^*+c)} = \frac{1}{2b^*}
\] Use Mills' ratio \[
\begin{align*}
&\frac{1}{b^*+c}\left(1 - \frac{1}{(b^*+c)^2}\right) \ge \frac{1}{2b^*} \ge \frac{1}{b^*+c}\\
\rightarrow&\frac{2(b^*+c)^2-2}{(b^*+c)^3}\ge \frac{1}{b^*} \ge \frac{2}{b^*+c}\\
\rightarrow&\frac{b^*+c}{2} \ge b^* \ge\frac{(b^*+c)^3}{2(b^*+c)^2-2}
\end{align*}
\] Thus boundary conditons for \(b^*\) is \[
c \le b^* \text{ and}\\
\frac{(b^*-c)(b^*+c)^2}{2} \le b^*
\]

    \subsection{(c) bisection method}\label{c-bisection-method}

    Let \(b^* = c+ \epsilon\), where \(\epsilon >0\) and \(c>0\) Then, \[
\begin{align*}
0 \le \frac{(b^*-c)(b^*+c)^2}{2} &= \frac{\epsilon (2c + \epsilon)^2}{2} \le c + \epsilon\\
&\rightarrow 0\le \epsilon(4c^2 + 4c\epsilon + \epsilon^2) \le 2c + 2\epsilon\\
&\rightarrow 0 \le 4\epsilon c^2 +4\epsilon^2c + \epsilon^3 \le 2c + 2\epsilon\\
&\rightarrow 0 \le 4 c^2 +4\epsilon c + \epsilon^2 \le 2\frac{c}{\epsilon} + 2\\
&\rightarrow 0 \le \frac{c}{\epsilon} +1 -2\epsilon c
\end{align*}
\] When \(0 \le \frac{c}{\epsilon} -2\epsilon c\),
\(\frac{c}{\epsilon} +1 -2\epsilon c\) always larger than 0 \[
\begin{align*}
0 &\le \frac{c}{\epsilon} -2\epsilon c\\
\rightarrow 0&\le \frac{1}{\epsilon} -2 \epsilon\\
\rightarrow 0 &\le \epsilon \le \frac{1}{\sqrt{2}}
\end{align*}
\] So, we can setting the upper bound of bisection as
\(c + \frac{1}{\sqrt{2}}\)

    \begin{Verbatim}[commandchars=\\\{\}]
{\color{incolor}In [{\color{incolor}14}]:} \PY{k}{def} \PY{n+nf}{bisection\PYZus{}test}\PY{p}{(}\PY{n}{b}\PY{p}{,}\PY{n}{c}\PY{p}{)}\PY{p}{:}
             \PY{k}{return}\PY{p}{(}\PY{n}{scipy}\PY{o}{.}\PY{n}{stats}\PY{o}{.}\PY{n}{norm}\PY{o}{.}\PY{n}{sf}\PY{p}{(}\PY{n}{b}\PY{o}{+}\PY{n}{c}\PY{p}{)}\PY{o}{/}\PY{n}{scipy}\PY{o}{.}\PY{n}{stats}\PY{o}{.}\PY{n}{norm}\PY{o}{.}\PY{n}{pdf}\PY{p}{(}\PY{n}{b}\PY{o}{+}\PY{n}{c}\PY{p}{)} \PY{o}{\PYZhy{}} \PY{l+m+mi}{1}\PY{o}{/}\PY{p}{(}\PY{l+m+mi}{2}\PY{o}{*}\PY{n}{b}\PY{p}{)}\PY{p}{)}    
\end{Verbatim}


    \begin{Verbatim}[commandchars=\\\{\}]
{\color{incolor}In [{\color{incolor}15}]:} \PY{k}{def} \PY{n+nf}{bisection}\PY{p}{(}\PY{n}{c}\PY{p}{,}\PY{n}{criteria} \PY{o}{=} \PY{l+m+mi}{10}\PY{o}{*}\PY{o}{*}\PY{p}{(}\PY{o}{\PYZhy{}}\PY{l+m+mi}{7}\PY{p}{)}\PY{p}{)}\PY{p}{:}
             \PY{n}{lower}\PY{o}{=} \PY{n}{c}
             \PY{n}{upper}\PY{o}{=} \PY{n}{c}\PY{o}{+}\PY{l+m+mi}{1}\PY{o}{/}\PY{n}{np}\PY{o}{.}\PY{n}{sqrt}\PY{p}{(}\PY{l+m+mi}{2}\PY{p}{)}
             \PY{k}{while} \PY{k+kc}{True}\PY{p}{:} 
                 \PY{n}{bstar} \PY{o}{=} \PY{p}{(}\PY{n}{lower}\PY{o}{+}\PY{n}{upper}\PY{p}{)}\PY{o}{/}\PY{l+m+mi}{2}
                 \PY{k}{if} \PY{n+nb}{abs}\PY{p}{(}\PY{n}{bisection\PYZus{}test}\PY{p}{(}\PY{n}{bstar}\PY{p}{,}\PY{n}{c}\PY{p}{)}\PY{p}{)}\PY{o}{\PYZlt{}}\PY{n}{criteria}\PY{p}{:}
                     \PY{k}{return}\PY{p}{(}\PY{n}{bstar}\PY{p}{)}
                     \PY{k}{break}
                 \PY{k}{elif} \PY{n}{bisection\PYZus{}test}\PY{p}{(}\PY{n}{lower}\PY{p}{,}\PY{n}{c}\PY{p}{)}\PY{o}{*}\PY{n}{bisection\PYZus{}test}\PY{p}{(}\PY{n}{bstar}\PY{p}{,}\PY{n}{c}\PY{p}{)}\PY{o}{\PYZlt{}}\PY{l+m+mi}{0}\PY{p}{:}
                     \PY{n}{upper} \PY{o}{=} \PY{n}{bstar}
                 \PY{k}{else}\PY{p}{:}
                     \PY{n}{lower} \PY{o}{=} \PY{n}{bstar}
\end{Verbatim}


    \begin{Verbatim}[commandchars=\\\{\}]
{\color{incolor}In [{\color{incolor}16}]:} \PY{k}{for} \PY{n}{i} \PY{o+ow}{in} \PY{p}{[}\PY{l+m+mi}{2}\PY{p}{,}\PY{l+m+mi}{3}\PY{p}{,}\PY{l+m+mi}{4}\PY{p}{]}\PY{p}{:}
             \PY{n+nb}{print}\PY{p}{(}\PY{l+s+s1}{\PYZsq{}}\PY{l+s+s1}{When c = }\PY{l+s+si}{\PYZpc{}d}\PY{l+s+s1}{, b* : }\PY{l+s+si}{\PYZpc{}f}\PY{l+s+s1}{\PYZsq{}} \PY{o}{\PYZpc{}}\PY{p}{(}\PY{n}{i}\PY{p}{,}\PY{n}{bisection}\PY{p}{(}\PY{n}{i}\PY{p}{)}\PY{p}{)}\PY{p}{)}
\end{Verbatim}


    \begin{Verbatim}[commandchars=\\\{\}]
When c = 2, b* : 2.215929
When c = 3, b* : 3.154852
When c = 4, b* : 4.119678

    \end{Verbatim}

    When \(c= 2,3\) and \(4\)\\
\(b^* = 2.215931,3.154846\) and \(4.119675\)

    \subsection{(d) Calculate the
efficiency}\label{d-calculate-the-efficiency}

We know that \$\hat{p} =\tilde{p}(0) \$.\\
From (a) \[
\begin{align*}
Var_g[\tilde{p}(b)] &= \frac{1}{n^2} \sum_{i=1}^{n} \left[ e^{b^2} S(c+b) - S^2(c)\right]\\
&=\frac{1}{n}\left[ e^{b^2} S(c+b) - S^2(c)\right]\\
\end{align*}
\] Thus, \[
Efficiency = \frac{Var[\hat{p}]}{Var[\tilde{p}(b^*)]} =\frac{ S(c) - S^2(c)}{e^{{b^*}^2} S(c+b^*) - S^2(c)}
\]

    \begin{Verbatim}[commandchars=\\\{\}]
{\color{incolor}In [{\color{incolor}17}]:} \PY{k}{def} \PY{n+nf}{varatio}\PY{p}{(}\PY{n}{b}\PY{p}{,}\PY{n}{c}\PY{p}{)}\PY{p}{:}
             \PY{n}{out} \PY{o}{=} \PY{p}{(}\PY{n}{scipy}\PY{o}{.}\PY{n}{stats}\PY{o}{.}\PY{n}{norm}\PY{o}{.}\PY{n}{sf}\PY{p}{(}\PY{n}{c}\PY{p}{)} \PY{o}{\PYZhy{}} 
                    \PY{n}{scipy}\PY{o}{.}\PY{n}{stats}\PY{o}{.}\PY{n}{norm}\PY{o}{.}\PY{n}{sf}\PY{p}{(}\PY{n}{c}\PY{p}{)}\PY{o}{*}\PY{o}{*}\PY{l+m+mi}{2}\PY{p}{)}\PY{o}{/} \PYZbs{}
                   \PY{p}{(}\PY{n}{np}\PY{o}{.}\PY{n}{exp}\PY{p}{(}\PY{n}{b}\PY{o}{*}\PY{o}{*}\PY{l+m+mi}{2}\PY{p}{)}\PY{o}{*}\PY{n}{scipy}\PY{o}{.}\PY{n}{stats}\PY{o}{.}\PY{n}{norm}\PY{o}{.}\PY{n}{sf}\PY{p}{(}\PY{n}{c}\PY{o}{+}\PY{n}{b}\PY{p}{)} 
                    \PY{o}{\PYZhy{}} \PY{n}{scipy}\PY{o}{.}\PY{n}{stats}\PY{o}{.}\PY{n}{norm}\PY{o}{.}\PY{n}{sf}\PY{p}{(}\PY{n}{c}\PY{p}{)}\PY{o}{*}\PY{o}{*}\PY{l+m+mi}{2}\PY{p}{)}
             \PY{k}{return}\PY{p}{(}\PY{n}{out}\PY{p}{)}
\end{Verbatim}


    \begin{Verbatim}[commandchars=\\\{\}]
{\color{incolor}In [{\color{incolor}18}]:} \PY{k}{for} \PY{n}{i} \PY{o+ow}{in} \PY{p}{[}\PY{l+m+mi}{2}\PY{p}{,}\PY{l+m+mi}{3}\PY{p}{,}\PY{l+m+mi}{4}\PY{p}{]}\PY{p}{:}
             \PY{n+nb}{print}\PY{p}{(}\PY{l+s+s1}{\PYZsq{}}\PY{l+s+s1}{Efficiency when c = }\PY{l+s+si}{\PYZpc{}d}\PY{l+s+s1}{ is : }\PY{l+s+si}{\PYZpc{}f}\PY{l+s+s1}{\PYZsq{}} \PY{o}{\PYZpc{}}\PY{p}{(}\PY{n}{i}\PY{p}{,}\PY{n}{varatio}\PY{p}{(}\PY{n}{bisection}\PY{p}{(}\PY{n}{i}\PY{p}{)}\PY{p}{,}\PY{n}{i}\PY{p}{)}\PY{p}{)}\PY{p}{)}
\end{Verbatim}


    \begin{Verbatim}[commandchars=\\\{\}]
Efficiency when c = 2 is : 19.001601
Efficiency when c = 3 is : 221.916501
Efficiency when c = 4 is : 7061.451247

    \end{Verbatim}

    When \(c= 2,3\) and \(4\)\\
\(Efficiency = 19.001601,221.916501\) and \(7061.451247\)


    % Add a bibliography block to the postdoc
    
    
    
    \end{document}
